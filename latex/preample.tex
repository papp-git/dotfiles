% LuaLateX------------------------------------------------------------------

%! TeX program=lualatex
%\usepackage{amsmath}
\usepackage{amssymb}
%\usepackage{fontspec}
\usepackage{unicode-math}   % lädt fontspec
\usepackage{luacode}
%\usepackage{lualatex-math}


% Sprache ------------------------------------------------------------

%\usepackage[T1]{fontenc} % nicht mit LuaLateX
\usepackage{lmodern}
\usepackage[ngerman]{babel}
\usepackage{hyphenat}
%\usepackage{parskip} % Konflikt mit KOMA -Script -> in documentclass

\setlength{\textheight}{720pt}
% -------------------------------------------------------------------

% Farben-----------------------------------------------------------
\usepackage[dvipsnames]{xcolor}
%---------------------------------------------------------------------
% TikZ--------------------------------------------------------------- 

\usepackage[siunitx]{circuitikz}
\ctikzset{european resistors, 
    bipoles/cuteswitch/thickness=0.25,
    }

\usetikzlibrary{arrows, shadows, positioning, shapes.symbols, shadings, patterns}

\usepackage{siunitx}
\DeclareSIUnit{\liter}{$\ell$}
\sisetup{
    locale = DE,
    per-mode = fraction,
    %exponent-to-prefix,
    %prefixes-as-symbols = false,
}

% Plots ---------------------------------------------------------------
\usepackage{pgfplots}
\pgfplotsset{compat=1.18}

% -----------------------\mathscr{ABC}-Font-Kalligraphie-----------------------------------
\usepackage{mathrsfs}


% tikz-Bilder importieren
% \includestandalone
\usepackage[mode=buildnew]{standalone}

\usepackage{graphicx}


% schickere Tabellen
\usepackage{booktabs}

% Mathe/Physik ------------------------------------------------------

\usepackage{cancel}
\renewcommand{\CancelColor}{\color{gray}}

\usepackage{esvect} %Vektorpfeile

%--------------------------------------------------------------------

% Pimp my document--------------------------------------------------

\usepackage{fancyhdr}
\usepackage{tasks}
\settasks{after-item-skip = 1ex, after-skip=2ex}
\usepackage{soul} % highlight
\usepackage[framemethod=TikZ]{mdframed}

%-------------------------------------------------------------------
\usepackage{tcolorbox}
\tcbuselibrary{skins}
%\usepackage{fontawesome} % veraltet
\usepackage{fontawesome5}
\usepackage{emoji}
% sonst------------------------------------------------------------

\usepackage{hyperref}
\usepackage{eurosym}

% ------------------------------------------------------------------

\newcommand{\kariert}[2]{
    \begin{tikzpicture}
    \draw[step=0.5cm,color=lightgray!60] (0,0) grid (#1 cm ,#2 cm);
    \end{tikzpicture}}

% exams
\usepackage{exsheets}
\SetupExSheets[question]{type=exam}
\SetupExSheets{solution/print=false}

% Punkt zu Komma 2.5 wird zu 2,5 
\ExplSyntaxOn
\cs_set_protected:Npn \exsheets_num:n #1
 {
  \num{\fp_eval:n{#1}}
 }
\ExplSyntaxOff

% Aufgabe box----------------------------------------------------
\newcounter{aufg}

\newtcolorbox{aufgabe}[1][\faPen]{                                              
    step=aufg,                         
    enhanced,
    colback=orange!10,                                                                  
    top=10pt, left=0pt, right=0pt,                                                   
    overlay={%
        \tikzset{overlay=false, anchor=west, rounded corners}
        \node[fill=orange!20, text=black,xshift=0.5cm,draw,      % background and textcolor of "Exercise" box
              minimum height=1.5em] at (frame.north west) (box1){\textbf{Aufgabe
              \theaufg}}; % position of "Exercise" box
        \node[text=orange, fill=white,draw,                % background and textcolor of difficulty rating box
              right = 3mm of box1, ] (difficulty){#1};     % position of difficulty rating box
    }}

%\newenvironment{aufgabe}[1][]{ 
%        \refstepcounter{aufg} 
%        \textbf{Aufgabe \theaufg} \hfill  #1} {\par} 

\newcounter{loesung}
\newenvironment{lsg}{ 
        \refstepcounter{loesung} 
        \textbf{Lösung \theloesung}} {}       
% -----------------------------------------------------------------

% Kästen ---------------------------------------------------------
\newenvironment{merke}[1][Merke]{
    \mdfsetup{
        frametitle={
            \tikz[baseline=(current bounding box.east), outer sep=0pt]
            \node[draw=red, anchor=east, rectangle, fill=red!20, rounded corners=3pt]
            {\strut #1};
                },
        innertopmargin=10pt, linecolor = red!20, linewidth=2pt, topline=true,
        roundcorner=5pt,
        backgroundcolor=red!5,
        frametitleaboveskip = \dimexpr-\ht\strutbox\relax,
        shadow=true,
        }
    \begin{mdframed} \relax }
    {\end{mdframed}}

%----------------------------------------------------------------    

\newenvironment{kasten}[1][]{

    \ifstrempty{#1}
        {\mdfsetup{roundcorner=5pt,
    backgroundcolor=lightgray!20,linecolor=gray, middlelinewidth=2pt,
    shadow=true,
    }}
    {\mdfsetup{
        frametitle={#1},
        linecolor=gray, middlelinewidth=2pt, frametitlerule=true,
        apptotikzsetting={\tikzset{mdfframetitlebackground/.append style={shade,left
        color=white, right color = lightgray!20}}},
        frametitlerulecolor=black!60,
        frametitlerulewidth=1pt,
        innertopmargin=\topskip,
        roundcorner=5pt,
        backgroundcolor=lightgray!20, shadow=true,
    }}
    \begin{mdframed}[]
    }
    { \end{mdframed} }

%------------------------------------------------------------------

\newenvironment{bsp}[2][Beispiel]{%}[gray]{

    \mdfsetup{
        frametitle={#1},
        linecolor=#2, 
        middlelinewidth=2pt, 
        apptotikzsetting={\tikzset{mdfframetitlebackground/.append style={shade,left
        color=#2, right color = #2!20}}},      
        frametitlerule=true,
        %frametitlebackgroundcolor=#2!80,
        frametitlerulecolor=#2!60,
        frametitlerulewidth=1pt,
        innertopmargin=\topskip,
        roundcorner=5pt,
        backgroundcolor=#2!20, 
        shadow=true,
    }
    \begin{mdframed}
    }
    { \end{mdframed} }


% ---------------------------------------------------------------

\newcommand{\onestar}{\color{orange} \faStar~\faStar[regular]~\faStar[regular]}
\newcommand{\twostar}{\color{orange}\faStar~\faStar~\faStar[regular]}
\newcommand{\threestar}{\color{orange}\faStar~\faStar~\faStar}
\newcommand{\onehalfstar}{\color{orange}\faStar~\faStarHalf*~\faStar[regular]}
\newcommand{\twohalfstar}{\color{orange}\faStar~\faStar~\faStarHalf*}

\newcommand{\versuch}{\faFlask~ Versuch:}
\newcommand{\durchfuehrung}{\faTools~ Durchführung:}
\newcommand{\beobachtung}{\faEye~ Beobachtung:}
\newcommand{\auswertung}{\faEdit~ Auswertung:}

% Zeichensätze
% -------------------------------------------------------------------

%\setmainfont{noto-serif}
%\setsansfont{Open Sans}
%\setmonofont{DejaVu Sans Mono}
%\setmathfont{Latin Modern Math}
