%
%   ██████╗ ██████╗ ███████╗ █████╗ ███╗   ███╗██████╗ ██╗     ███████╗
%   ██╔══██╗██╔══██╗██╔════╝██╔══██╗████╗ ████║██╔══██╗██║     ██╔════╝
%   ██████╔╝██████╔╝█████╗  ███████║██╔████╔██║██████╔╝██║     █████╗  
%   ██╔═══╝ ██╔══██╗██╔══╝  ██╔══██║██║╚██╔╝██║██╔═══╝ ██║     ██╔══╝  
%   ██║     ██║  ██║███████╗██║  ██║██║ ╚═╝ ██║██║     ███████╗███████╗
%   ╚═╝     ╚═╝  ╚═╝╚══════╝╚═╝  ╚═╝╚═╝     ╚═╝╚═╝     ╚══════╝╚══════╝
                                                                   
% LuaLateX------------------------------------------------------------------

%! TeX program=lualatex

% Mathe 
\usepackage{mathtools}
\usepackage{amssymb}
\usepackage{mathrsfs} % \mathscr{ABC}-Font-Kalligraphie
\usepackage{cancel}
    \renewcommand{\CancelColor}{\color{gray}}
\usepackage{esvect} %Vektorpfeile

\usepackage{marvosym} % Eurosymbol  

% Sprache ------------------------------------------------------------
\usepackage{lmodern}
\usepackage[ngerman]{babel}
\usepackage{hyphenat}

% Farben-----------------------------------------------------------
\usepackage[dvipsnames, table]{xcolor}

% TikZ--------------------------------------------------------------- 
\usepackage[siunitx]{circuitikz}
    \ctikzset{european resistors, 
	bipoles/cuteswitch/thickness=0.25,
	}

\tikzset{
	achse/.style={very thick, -latex},
	koor/.style={color=lightgray!60,step=0.5},
    }

\usetikzlibrary{arrows, shadows, positioning, shapes.symbols, shadings, patterns, intersections}

\DeclareSIUnit{\liter}{$\ell$}
    \sisetup{
	locale = DE,
	per-mode = fraction,
    }

% Plots ---------------------------------------------------------------
\usepackage{pgfplots}
\pgfplotsset{compat=1.18}

% tikz-Bilder importieren
% \includestandalone
\usepackage[mode=buildnew]{standalone}

\usepackage{graphicx}

% schickere Tabellen
\usepackage{booktabs}

% Pimp my document--------------------------------------------------
\usepackage{fancyhdr}
\usepackage{tasks}
    \settasks{after-item-skip = 1ex, after-skip=2ex}
\usepackage{soul} % highlight
\setlength{\textheight}{720pt}

% Boxen--------------------------------------------------------------
\usepackage[framemethod=TikZ]{mdframed}

\usepackage[most]{tcolorbox}
    \tcbuselibrary{skins}

% -------------------icons/emoji -----------------------------------
\usepackage{fontawesome5}
\usepackage{emoji}

% sonst------------------------------------------------------------
\usepackage{hyperref}
\usepackage{eurosym}

% -------------------Kästchen------------------------------------------
\newcommand{\kariert}[2]{
    \begin{tikzpicture}
    \draw[step=0.5cm,color=lightgray!60] (0,0) grid (#1 cm ,#2 cm);
    \end{tikzpicture}}

% exams--------------------------------------------------------
\usepackage{exsheets}
    \SetupExSheets[question]{type=exam}
    \SetupExSheets{solution/print=false}

    % Punkt zu Komma 2.5 wird zu 2,5 
    \ExplSyntaxOn
    \cs_set_protected:Npn \exsheets_num:n #1
    {
	\num{\fp_eval:n{#1}}
    }
    \ExplSyntaxOff

%   :::    :::     :::      :::::::: ::::::::::: :::::::::: ::::    ::: 
%   :+:   :+:    :+: :+:   :+:    :+:    :+:     :+:        :+:+:   :+: 
%   +:+  +:+    +:+   +:+  +:+           +:+     +:+        :+:+:+  +:+ 
%   +#++:++    +#++:++#++: +#++:++#++    +#+     +#++:++#   +#+ +:+ +#+ 
%   +#+  +#+   +#+     +#+        +#+    +#+     +#+        +#+  +#+#+# 
%   #+#   #+#  #+#     #+# #+#    #+#    #+#     #+#        #+#   #+#+# 
%   ###    ### ###     ###  ########     ###     ########## ###    #### 

% Aufgabe box----------------------------------------------------
\newcounter{aufg}

\newtcolorbox{aufgabe}[1][\faPen]{                                              
    step=aufg,                         
    enhanced,
    colback=orange!10,                                                                  
    top=10pt, left=0pt, right=0pt,                                                   
    overlay={%
        \tikzset{overlay=false, anchor=west, rounded corners}
        \node[fill=orange!20, text=black,xshift=0.5cm,draw,      
	% background and textcolor of "Exercise" box
              minimum height=1.5em] at (frame.north west) (box1){\textbf{Aufgabe
              \theaufg}}; % position of "Exercise" box
        \node[text=orange, fill=white,draw,                
	% background and textcolor of difficulty rating box
              right = 3mm of box1, ] (difficulty){#1};     
	% position of difficulty rating box
    }}

\newcounter{loesung}
\newenvironment{lsg}{ 
        \refstepcounter{loesung} 
        \textbf{Lösung \theloesung}} {}       

% Merke ---------------------------------------------------------
\newenvironment{merke}[1][Merke]{
    \mdfsetup{
        frametitle={
            \tikz[baseline=(current bounding box.east), outer sep=0pt]
            \node[draw=red, anchor=east, rectangle, fill=red!20, rounded corners=3pt]
            {\strut #1};
                },
        innertopmargin=10pt, linecolor = red!20, linewidth=2pt, topline=true,
        roundcorner=5pt,
        backgroundcolor=red!5,
        frametitleaboveskip = \dimexpr-\ht\strutbox\relax,
        shadow=true,
        }
    \begin{mdframed} \relax }
    {\end{mdframed}}

% Kasten ---------------------------------------------------------------    

\makeatletter
\newtcolorbox{kasten}[2][]{%
enhanced, 
breakable,
colback=white,
colframe= cyan!50!black,
attach boxed title to top left={yshift=-2pt}, title={\textbf{#2}},
boxed title size=standard,
boxrule=0pt,
boxed title style={%
    sharp corners, 
    rounded corners=northwest, 
    colback=tcbcolframe, 
    boxrule=0pt},
sharp corners=north,
overlay unbroken={%
    \path[fill=tcbcolback] 
        ([xshift=-2pt]title.south east) 
        to[out=0, in=180] ([xshift=1.5cm]title.east)--
        (title.east-|frame.east) |- 
        ([xshift=-2pt]title.south east)--cycle;
    \path[fill=tcbcolframe] (title.south east) 
        to[out=0, in=180] ([xshift=1.5cm]title.east)--
        (title.east-|frame.east)
        [rounded corners=\kvtcb@arc] |- 
        (title.north-|frame.north) 
        [sharp corners] -| (title.south east);
    \draw[line width=.5mm, rounded corners=\kvtcb@arc, 
        tcbcolframe] 
        (title.north west) rectangle 
        (frame.south east);
}, 
overlay first={%
    \path[fill=tcbcolback] 
        ([xshift=-2pt]title.south east) 
        to[out=0, in=180] ([xshift=1.5cm]title.east)--
        (title.east-|frame.east) |- 
        ([xshift=-2pt]title.south east)--cycle;
    \path[fill=tcbcolframe] (title.south east) 
        to[out=0, in=180] ([xshift=1.5cm]title.east)--
        (title.east-|frame.east)
        [rounded corners=\kvtcb@arc] |- 
        (title.north-|frame.north) 
        [sharp corners] -| (title.south east);
    \draw[line width=.5mm, rounded corners=\kvtcb@arc, 
        tcbcolframe] 
        (frame.south west) |- (title.north) -| 
        (frame.south east);
}, 
overlay middle={%
    \draw[line width=.5mm, tcbcolframe] 
    (frame.north west)--(frame.south west) 
    (frame.north east)--(frame.south east);
}, 
overlay last={%
    \draw[line width=.5mm, rounded corners=\kvtcb@arc, 
        tcbcolframe] 
        (frame.north west) |- (frame.south) -|
        (frame.north east);
}, 
#1
}
\makeatother

% Beispiel----------------------------------------------------------------

\newenvironment{bsp}[2][Beispiel]{%}[gray]{

    \mdfsetup{
        frametitle={#1},
        linecolor=#2, 
        middlelinewidth=2pt, 
        apptotikzsetting={\tikzset{mdfframetitlebackground/.append style={shade,left
        color=#2, right color = #2!20}}},      
        frametitlerule=true,
        %frametitlebackgroundcolor=#2!80,
        frametitlerulecolor=#2!60,
        frametitlerulewidth=1pt,
        innertopmargin=\topskip,
        roundcorner=5pt,
        backgroundcolor=#2!20, 
        shadow=true,
    }
    \begin{mdframed}
    }
    { \end{mdframed} }


% Übung ---------------------------------------------------------------

\newenvironment{uebung}[1]{%
    \begin{tikzpicture}%[#1]%
        \def\uebungname{#1}%
        \node [draw,inner sep=1.5ex,text width=\textwidth, fill=Cerulean!5, drop shadow, 
	    rounded corners]% good options: minimum height, minimum width
            (BOXCONTENT) \bgroup\rule{0pt}{3ex}\ignorespaces
}{%
        \egroup;
        \node [right,inner xsep=1em,fill=Cerulean!75,outer sep=0pt,text height=2ex,text depth=.5ex] (BOXNAME) 
	    at ([shift={(-1em,0pt)}]BOXCONTENT.north west) {\large \textsc{\uebungname}};
        \fill[Cerulean] (BOXNAME.north east) -- +(-1em,1em) -- +(-1em,0) -- cycle;
        \fill[Cerulean] (BOXNAME.south west) -- +(1em,-1em) -- +(1em,0) -- cycle;
    \end{tikzpicture}
}

% Banner --------------------------------------------------------------
\newcommand{\ueben}[1][Übungen]{
    \begin{figure}[!h]
	\centering
	\huge
	\begin{tikzpicture}
	    \node [right,inner xsep=5em,fill=Cerulean!75,outer sep=0pt,
		text height=2ex,text depth=.5ex] (BOXNAME) 
		{\emoji{abacus}~\textsc{#1}~ 
		\emoji{woman-technologist-medium-skin-tone}};
	    \fill[Cerulean] (BOXNAME.north east) -- +(-1em,1em) -- +(-1em,0) -- cycle;
	    \fill[Cerulean] (BOXNAME.south west) -- +(1em,-1em) -- +(1em,0) -- cycle;
	\end{tikzpicture}
\end{figure}
}

% Banner ------------------------------------------------------------------
\newcommand{\thema}[2]{
\pgfmathsetmacro{\myscale}{\textwidth/16.85cm}%
\begin{tikzpicture}[nodes={inner sep=0pt,outer sep=0pt},scale=\myscale]
 \fill[violet!90, rounded corners]  (0,0) rectangle (18.85,3.25);
 \fill[violet!30] (10,0) to[out=14,in=-140] (17.32,3.25)--
  (17.6,3.25) to[out=-140,in=15] (11.7,0) -- cycle;
 \fill[violet!65, rounded corners] (17.6,3.25) to[out=-140,in=15] (11.7,0) -| (18.85,3.25) -- cycle;
 \node[anchor=south west,text=white,font=\sffamily\bfseries,scale=3*\myscale] 
     at (0.5,1.5)
     {#1};
 \node[anchor=south east,text=white,font=\sffamily\bfseries,scale=2*\myscale] 
     at (18.5,0.2) {#2};
\end{tikzpicture}
}

% Sterne------------------------------------------------------------------
\newcommand{\onestar}{\color{orange} \faStar~\faStar[regular]~\faStar[regular]}
\newcommand{\twostar}{\color{orange}\faStar~\faStar~\faStar[regular]}
\newcommand{\threestar}{\color{orange}\faStar~\faStar~\faStar}
\newcommand{\onehalfstar}{\color{orange}\faStar~\faStarHalf*~\faStar[regular]}
\newcommand{\twohalfstar}{\color{orange}\faStar~\faStar~\faStarHalf*}

% Versuch 
\newcommand{\versuch}{\faFlask~ Versuch:}
\newcommand{\durchfuehrung}{\faTools~ Durchführung:}
\newcommand{\beobachtung}{\faEye~ Beobachtung:}
\newcommand{\auswertung}{\faEdit~ Auswertung:}

% Zeichensätze

%\setmainfont{noto-serif}
%\setsansfont{Open Sans}
%\setmonofont{DejaVu Sans Mono}
%\setmathfont{Latin Modern Math}
